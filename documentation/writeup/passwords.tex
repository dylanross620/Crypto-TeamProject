\section{Password Management}\label{sec:passwords}
Done by Dylan Ross\\

	\medskip
	In the project we were given to attack, the first vulnerability that we found became apparent immediately upon reading the documentation. The code does not make use of passwords of any
	kind, and instead allows the ATM to check the balance of, deposit into, and withdraw from arbitrary accounts without authentication. The could trivially be leveraged by an attacker
	to steal money from other people's accounts by selecting a different account during a withdrawal.\\

	It could be said that this attack is outside of the scope of attacking the cryptographic communication between an ATM and a bank as the ATM could be handling user authentication locally, but
	that would likely not be feasible. In order for that scheme to work, every ATM would need to locally store the credentials of every user in the bank, which would require a significant amount
	of memory and as such would drastically increase the price of the ATM machines themselves. Additionally, this scheme would require updating the local storage of every ATM owned by the bank
	whenever a new user account is created, which would cost a lot of time and money.
