\section{Connection Sequence Weaknesses}\label{sec:conseq}
Done by Ryan Prashad\\

	\medskip
	In the provided blackhat code, the other team uses an attribute of every connection labeled the sequence in order to prevent replay attacks. This method works by incrementing a counter for each message sent to verify that messages are being sent in order and are not being taken out of context to randomly send to the bank in a future session to force unexpected behavior.
	
	\subsection{Connection Struct}\label{sec:connstruct}
		The connection struct in this code basically carries all the applicable encryption schemes locally for the bank and atm. The agreed upon suite in the handshake is imported into this struct and then used for all the send/receive calls. The sequence attribute of the \texttt{ssh\_t} struct is the counter that gets incremented from message to message in this code between the bank and atm. Each time the recv or send commands from the ssh call are used, the connection sequence variable is correctly incremented by the pad/unpad function.
	
	\subsection{Sequence Vunerabilities}\label{sec:seqvun}
		The way that this code is written, connection.sequence is there but does not guard against injecting the same message twice to the bank server within a single session. If there was a man in the middle between the bank and atm and they intercepted the ciphertext, the ciphertext could be injected back into the bank receive call to do the same action again. An example of this could be draining someones bank account by sending the same false withdraw statement over and over again, or depositing money into your own bank account by interception your own message and sending it to the bank to infinitely deposit money there.
	
	\subsection{Using the Vunerability}\label{sec:usevun}
		Because the ssh class never asserts that connection.sequence has any order, sending the same message twice could technically work because the encryption is still valid in this session, and sequence is never checked against the previous sequence to see if the message is in the right order (of increasing sequence).
	
	\subsection{Code to take advantage}\label{sec:takeadv}
		Modified atm code found in seqexploit.c assumes we've intercepted a message going to the bank from the atm accross localhost already. Since this message is already encrypted, we represent the input to this code as a string of 2 digit hex numbers. This code is then injected by just sending it to the bank (even though we are not an atm at this point). If it works, the bank should handle this message as normal and proceed and we've succeeded with injection.
		
		\subsection{Limitations}\label{sec:lims}
			Due to the way the bank file was written, we cannot actually connect an atm and our exploit code to the bank server. If the bank.c file were setup to take a  variable amount of connections, we can spoof being an atm or just send the intercepted code to the bank and trigger the recieve call on its port. The exploit code as is just interprets the hex encoded representation to an unsigned character array and sends the message as if it were an atm.