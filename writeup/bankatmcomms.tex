\section{Bank/ATM Communication}\label{sec:batmcoms}
Made by all three members\\
Writeup by Ryan Prashad\\

	\medskip
	The whole scheme of Bank/ATM communication relies on AES symmetric encryption and HMAC after public key encryption is used to  establish the symmetric keys for use by way of Diffie-Hellman. This allows us to send arbitrary length messages after the handshake between the ATM and Bank without the need to refer to a modulus n-bits and cap the messsage size at that (as would be the case with asymmetric encryption).
	
	\subsection{AES Communication}\label{sec:aescoms}
		The communication between the Bank and ATM is concatenated with the dash character in order to have a smooth time parsing on either end of the receive. The AES implementation allows us to concatenate and fit a wide variety of messages/return status all in a single encryption call. AES is used for all communication between the Bank and ATM after the Diffie-Hellman key exchange is successful on both ends.
	
	\subsection{HMAC Communication}\label{sec:hmaccoms}
		An HMAC implementation is used in order to verify all encrypted AES messages after sending over the network being received. Since AES encryption allows for a variable length string encryption, we can append an HMAC of the sent message onto the actual message before AES encryption. This makes it so that the HMAC is protected by its exchanged public MAC key that was transferred over the handshake, and the actual encrypted message is safe from tampering. If for some reason after decryption the HMAC of the plaintext does not equal the appended HMAC that was send over AES encryption alongside the now plaintext, we can reject the message. This method also provides forward secrecy that lies in HMAC key generation. Since each key is randomly generated by the Diffie-Hellman key exchange, trying to verify the same message with the same HMAC will not work because the HMAC keys will not match and thus the server will reject this message as being tampered with.
