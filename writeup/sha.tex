\section{SHA1}\label{sec:sha}
Done by Dylan Ross\\

	\medskip
	SHA-1 is a cryptographic hashing function that was first published in 1995. It works by padding messages to specific lengths and doing various
	computations on pieces of the message at a time to create a fixed-length unique output for each input, regardless of the input size. We implemented the SHA-1 algorithm according to 
	\href{https://nvlpubs.nist.gov/nistpubs/FIPS/NIST.FIPS.180-4.pdf}{FIPS 180}.

	\subsection{SHA-1 Functions}\label{sec:sha-functions}
		Each version of SHA has its own function definitions. In all cases, the function takes 3 words as inputs, and returns a different result depending on which iteration of the function
		is being called which is referred to as $t$. Using the symbol $\oplus$ for the XOR operation, $\land$ for the bitwise and operation, and $\neg$ for the bitwise negation operation, 
		$f_t(x,y,z)$ where each of $x,y,z$ is a 32-bit word, SHA-1 defines its functions as follows:
		\begin{center}
			$f_t(x,y,z) = \begin{cases}
				(x\land y) \oplus (\neg x\land z) & 0\le t\le19\\
				x\oplus y\oplus z & (20\le t\le39) \lor (60\le t\le79)\\
				(x\land y) \oplus (x\land z) \oplus (y\land z) & 40\le t\le59
				\end{cases}	 
			$
		\end{center}   

	\subsection{SHA-1 Constants}\label{sec:sha-constants}
		In addition to defining their own functions, each version of SHA also defines its own constants. SHA-1 defines eighty words as constants that it treats as an array called $K_t$. Similarly
		to the SHA-1 functions, the values of $K_t$ are one of four possible values, where each value appears twenty consecutive times.

	\subsection{Padding}\label{sec:sha-padding}
		In order to work on inputs of any length (to a point), SHA-1 pads messages until their bitlength is a multiple of 512. This is important as the SHA-1 algorithm works by processing the
		message in blocks of 512 bits at a time, so the entire message must have a whole number of blocks. This is done by first appending a 1 bit to the message, followed by the minimum number
		of 0 bits such that the new bitlength $b\equiv448\text{ mod }512$. To finish the padding, 64 bits are appended to the end, where the 64 bits are the binary representation of the
		original message size.

	\subsection{Calculating the Hash}\label{sec:sha-hash}
		To compute a message's hash, first pad the message using the padding algorithm described in section \ref{sec:sha-padding}. Additionally, 5 initial hash values must be defined, which are given
		in \href{https://nvlpubs.nist.gov/nistpubs/FIPS/NIST.FIPS.180-4.pdf}{FIPS 180}. At this point, the message is split into 512-bit blocks and each block is worked on seperately.\\

		For each block, a message schedule is created by either copying a word from the original message or manipulating previous parts of the message schedule, depending on which part of the
		schedule is being calculated. From there, the current hash values are modified through a combination of addition, the functions defined in section \ref{sec:sha-functions}, and circular
		left shifts. Within the SHA-1 algorithm, all addition is done mod $2^{32}$ to maintain each part fitting within a word.\\

		After the final block has been hashed, the message's resultant hash is computed by concatenating the 5 resultant hash values that have been modified by every block.

	\subsection{Implementation}\label{sec:sha-implementation}
		When it came to implementing the SHA-1 algorithm, the process was fairly straightforward. The entire process was a matter of implementing three small functions, having
		logic to determine which one to execute, and initializing constants. The padding algorithm was also easy to implement as it involved simple modular arithmetic to calculate padding size.
		However, an \href{https://csrc.nist.gov/CSRC/media/Projects/Cryptographic-Standards-and-Guidelines/documents/examples/SHA1.pdf}{example hash calculation} provided by NIST was very
		helpful in ensuring that my algorithm was working as intended at every step.
