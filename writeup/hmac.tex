\section{HMAC}\label{sec:hmac}
Done by Dylan Ross\\

	\medskip
	HMAC is a standard for using a cryptographic hash function (such as SHA-1 as defined in section \ref{sec:sha}) in order to create a keyed message authentication code. The standard treats the
	chosen hash function as an oracle, and as such is very simple and easy to implement. We implemented the HMAC algorithm according to \href{https://tools.ietf.org/html/rfc2104}{RFC 2104}.

	\subsection{MAC Generation}\label{sec:hmac-generation}
		We will define $H(m)$ as a cryptographic hash function to be used by HMAC, and $B$ to be the block length of $H(m)$ in bytes.\\

		Firstly, the HMAC key must be $B$ bytes long, so some key manipulation is likely needed. If the key $K$ is longer than $B$ bytes long, then replace $K$ with $H(K)$. Next, if $K$ is less
		than $B$ bytes long, then $K$ is padded by appending null bytes (0x00) until $K$ has exactly $B$ bytes.

		We will define $ipad$ as the byte 0x36 repeated $B$ times, and $opad$ as the byte 0x5c repeated $B$ times. Using the symbol $\oplus$ to represent the XOR operation and $+$ as string
		concatenation, the HMAC of message $m$ is computed as $HMAC=H(K\oplus opad + H(K\oplus ipad + m))$.

	\subsection{Implementation}\label{sec:hmac-implementation}
		The implementation of HMAC is very simple, given an already working hash function. As such, we used the SHA-1 algorithm discussed in section \ref{sec:sha} as our hash function for
		generating HMACs. The HMAC algorithm can be made more efficient by utilizing a map data structure. Because $opad$ and $ipad$ are XORed with the key and used for nothing else, each
		key can have the values $K\oplus ipad$ and $K\oplus opad$ calculated the first time the key is used and retrieved by future uses of the same key. This way, a relatively large calculation
		can be skipped many times during long communications that involve many packets being sent, each with its own MAC.
