\section{RSA Encryption Scheme}\label{sec:rsa}
Done by Dylan Ross\\

\medskip
	In our project, we chose to implement the RSA public encryption scheme. This scheme has been around since the 1970s when it was invented by Ron Rivest, Adi Shamir, and Leonard Adleman.
	The general idea of the scheme is to use the factorization of a large composite integer as a trapdoor function in order to be able to reverse encryption that was done via modular
	exponentiation.

	\subsection{Background Math}\label{sec:rsa-background}
		This encryption scheme relies heavily on modular arithmetic, notably exponentiation. Below are necessary sections to understand the math behind the RSA algorithm.

		\subsubsection{GCD, Prime Numbers, and Coprimality}\label{sec:rsa-gcd}
			A prime number is an integer greater than 1 that has exactly 2 factors: 1 and itself. Examples of such numbers include 2, 3, 5, 7, and 11. The exact distribution of prime numbers
			is not known, and this fact plays a large role in various parts of cryptography. However, the distribution of prime numbers can be approximated by the prime number theorem, which
			states that $\pi(N)\sim\frac{N}{\text{log}(N)}$ where $\pi(N)$ is the number of prime numbers $p$ such that $p\leq N$. Every integer can be written as a product of primes, but
			finding such primes is a difficult problem. This problem is what creates the security of RSA.\\

			We will define $gcd(a, b)$ as the greatest common divisor of integers $a$ and $b$. That is, $gcd(a, b)$ is the largest integer $g$ such that $g|a$ and $g|b$. For example, 
			$gcd(4,6)=2$ as the factors of 4 are $\{1,2,4\}$, the factors of 6 are $\{1,2,3,6\}$ and 2 is the largest integer in the intersection of these factors. We say that two numbers
			are coprime if their $gcd$ is 1. For example, 4 and 9 are coprime as they share no factors besides 1 even though neither number is prime. By the definition of prime numbers, a
			prime number $p$ is coprime with all numbers $a$ when $a<p$. That is, $\forall a<p, p\in\text{prime numbers}:gcd(a, p)=1$. The $gcd$ of two integers $a$ and $b$ can efficiently
			be computed via the Euclidean algorithm, which states that $gcd(a,b)=gcd(b,a\text{ mod }b)$.

		\subsubsection{Fermat's Little Theorem and Primality Testing}\label{sec:rsa-fermat}
			Fermat's little theorem states that, for any prime $p$ and any integer $a$ such that $a\not\equiv0\text{ mod }p$, $a^{p-1}\equiv1\text{ mod }p$. While deterministic primality
			testing is infeasible due to the unknown distribution of primes, Fermat's little theorem provides an efficient way to perform a probabilistic primality test. For any number
			$n$, we can test if $n$ is prime by generating a random number $a$ where $1\leq a<n$ and test if $a^{n-1}\equiv1\text{ mod }n$. If this test does not hold, $n$ is definitely
			composite, while the test holding implies that $n$ may be prime. Repeating this test several times can greatly reduce the probability of incorrectly labelling a composite as
			prime.\\

			Using the primality test described above and the approximate distribution of primes discused in section \ref{sec:rsa-gcd}, we can efficiently generate random primes of a desired bit
			length. This can be done by generating random numbers of the correct length and testing if they are prime. Due to the approximation of $\pi{N}$, we can expect this process to
			be approximately linear in time complexity.

		\subsubsection{Euler's Theorem}\label{sec:rsa-eulers}
			We will define Euler's totient function $\phi(n)$ as the number of integers $a<n$ such that $gcd(a, n)=1$. In the case of a prime $p$, as was mentioned in section \ref{sec:rsa-gcd},
			$\phi(p)=p-1$ and as such Euler's theorem is consistent with Fermat's little theorem. Given a composite number $n=pq$ where $p$ and $q$ are primes, we can calculate the totient of
			$n$ by $\phi(n)=\phi(p)*\phi(q)=(p-1)*(q-1)$. As such, we can calculate the totient of a composite number in constant time if we have its prime factorization. This fact serves as 
			the basis of the trapdoor function of RSA.\\

			Euler's Theorem states that, for $a$ and $n$ such that $gcd(a,n)=1$, $a^{\phi(n)}\equiv 1\text{ mod }n$. From here, it follow that $a^{\phi(n)+1}\equiv a\text{ mod }n$. This is
			because $a^{\phi(n)+1}=a^{\phi(n)}*a\equiv 1*a=1\text{ mod }n$.

	\subsection{Key Generation}\label{sec:rsa-keys}
		RSA keys have two parts: the public key and the private key. The public key consists of the integers $N$ and $e$ and is used for message encryption while the private key consists of the
		integers $N$ and $d$ and is used for message decryption. As the names imply, the public key can be seen by anyone while the private key is kept as a secret.\\

		To generate the keys, first pick two large primes $p$ and $q$ that are a similar size. We will let $N=pq$ be the public (and private) modulus. We will also pick a random integer $e$ such
		that $e<\phi(N)$ and $gcd(e,\phi(N))=1$. As was mentioned above, the public key consists of this pair ($N$, $e$). We then calculate $d\equiv e^{-1}\text{ mod }\phi(N)$, and this makes up
		the private key along with $N$.

	\subsection{Encryption}\label{sec:rsa-encryption}
		In order to send an encrypted message using RSA, the sender will encrypt the message using the intended receiver's public key. As such, anyone can encrypt and send a person a message,
		but only the intended recipient can decrypt and read the message using the corresponding private key.\\

		Recall that the receiver's public key is made up of two parts, $N$ and $e$. First, the sender encrypts their message as an integer $m$ where $m<N$. They can then calculate the ciphertext
		$c=m^e\text{ mod }N$. This ciphertext can then be sent to the recipient to be decrypted and read.

	\subsection{Decryption}\label{sec:rsa-decryption}
		In order to decrypt a message that was encrypted using RSA, the receiver must use their own private key. Recall that the private key consists of two parts, $N$ and $d$. Upon receiving
		ciphertext $c$, the message can be decrypted to the original message $m$ by computing $m=c^d\text{ mod }N$. This works due to the construction of $e$ and $d$. Because $d\equiv e^{-1}
		\text{ mod }\phi(N)$, we know that $de=k\phi(n)+1$ for some integer $k$. Due to the fact that $c\equiv m^e\text{ mod }N$, $c^d\equiv (m^e)^d=m^{ed}=m^{k\phi(N)+1}$. It follows from the
		results of section \ref{sec:rsa-eulers} that this is equivalent to the original message $m$.

	\subsection{Message Signing}\label{sec:rsa-sign}
		In addition to encrypting messages to be sent to others, RSA can also be used to generate a signature for a message. A signature is a message that is encrypted in some way using a user's
		private key that can be verified using their public key in order to prove the user's identity.\\

		Because RSA encrypts and decrypts via exponentiation, the scheme is commutative. That is because, for all $m$, $e$, and $d$, $(m^e)^d=(m^d)^e$. As such, a message can be signed by
		encrypting it using your own private key. Alternatively, a signature for a message can be generated by doing the aforementioned private key encryption on the hash of the message. This
		type of signature can be verified by computing the hash of the message and checking that it equals $s^e\text{ mod }p$ where $s$ is the signature.

	\subsection{Security Measures in Implementation}\label{sec:rsa-security}
		RSA has many pitfalls that can result in a loss of security if implemented improperly. For instance, there are an entire class of attacks that stem from the public exponent $e$ being
		too small. Selecting $e=3$ is a common example of this, and the decreased size of $e$ makes the encryption process faster. However, it allows attacks such as Hastad's Broadcast attack,
		where the same message being sent to more than $e$ people can be decrypted without knowing $d$ by using the Chinese Remainder Theorem. Even more simply, it is very important that
		$m^e\ge N$. If $e$ and $m$ are small enough that this is not true, then the message can be decrypted by calculating $m=\sqrt[e]c$. Another potential pitfall of RSA is if the private 
		exponent $d$ is too small. In cases where $d < \frac13N^{\frac14}$, Wiener's Attack can be used to recover $d$ and break the encryption using Wiener's theorem and continued fractions.\\

		Even in the case that the parameters are chosen appropriately, some vulnerabilities may remain due to the malleable nature of RSA. Given ciphertext $c\equiv m^e$, an attacker can choose
		a desired constant $r$ and compute a new ciphertext $c'$ that decrypts to $m'=rm$. This is done by calculating $c'=r^ec\text{ mod }N$. This works because, during the decryption process,
		the receiver will compute $m'\equiv (c')^d=(r^ec)^d=(r^em^e)^d=r^{ed}m^{ed}\text{ mod }N\equiv rm$.\\

		During our implementation of RSA, we took all of these potential vulnerabilities into account. In order to prevent small public exponent attacks, we set a fixed value for $e$ of 65537.
		This value of $e$ is large enough to prevent small public exponent attacks, yet it a prime number that can be written as $2^k+1$. The fact that $e$ is prime is important since $e$ must
		be coprime with $\phi(N)$ and a prime $e$ makes this much more likely. The fact that $e$ is one more than a power of 2 means that the exponentiation of $m^e\text{ mod }N$ can be
		computed quickly as well. In order to facilitate using a chosen value for $e$, we had to add some additional logic when generating $p$ and $q$. We only accepted values such that $p-1$ and
		$q-1$ were coprime with our desired value of $e$. Because $\phi(N)=(p-1)*(q-1)$, if both factors were coprime to $e$ then $\phi$ must also be coprime. In order to prevent small private
		exponent attacks, we added a check during key generation to see if the resulting $d$ was too small (see Wiener's attack bound above). In cases where $d$ is too small, we restart the key
		generation. In order to prevent our ciphertexts from being malleable, each one also contains a hash of the message. As such, each ciphertext is plaintext aware, and multiplying the
		ciphertext by a constant will prevent the hash from matching and alert the receiver that the message has been tampered with. In order to ensure that we did not make any errors in our
		RSA implementation, we made sure to reference the paper \href{https://crypto.stanford.edu/~dabo/pubs/papers/RSA-survey.pdf}{"Twenty Years of Attacks on the RSA Cryptosystem"} by Dan
		Boneh to check if anything mentioned in the paper applied to our system.
