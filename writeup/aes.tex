\section{AES Encryption Scheme}\label{sec:aes}
Done by Dylan Ross\\

	\medskip
	We implemented AES according to \href{https://nvlpubs.nist.gov/nistpubs/FIPS/NIST.FIPS.197.pdf}{FIPS 197}. AES is a private key cryptosystem that was designed to replace the previously
	standard DES encryption algorithm. AES works by splitting the message into blocks of 16 bytes and encrypting each block seperately.

	\subsection{Mathematical Definitions}\label{sec:aes-math}
		AES treats each byte as a polynomial with its bits as the coefficients. This polynomial representation results in the byte 0x55, with binary representation 0101 0101 being treated as 
		the polynomial $x^6+x^4+x^2+x^0$.

		\subsubsection{Addition}\label{sec:aes-add}
			Within AES, adding bytes is done by adding their representative polynomials mod 2. On a bit level, this is the equivalent of the bitwise XOR operation, which will be represented by
			the symbol $\oplus$. Additionally, this means that addition and subtraction are the same, as XOR is its own inverse.

		\subsubsection{Multiplication}\label{sec:aes-mult}
			Within AES, multiplication is done in GF($2^8$) and reduced mod an irreducible polynomial $m$, which is defined as $m(x)=x^8+x^4+x^3+x+1$. This multiplication is represented by the
			$\bullet$ symbol.\\

			This multiplication can be drastically simplified using bitwise operations. Firstly, the byte 0x01 is the identity byte, so $x\bullet0x01=x$ for all bytes $x$. In order to multiply
			a byte $b$ by 0x02, first you must take note of the most significant bit of $b$. To calculate the result, you then shift $b$ left by 1 bit, then XORing it with 0x1b if the most
			significant bit was a 1 (you only do the shift if the bit was a 0). As such, the result when multiplying is $b<<1$ if the most significant bit is 0, else $(b<<1)\oplus\text{0x1b}$.
			Applying this formula multiple times allows multiplying by any power of 2, which can be combined with the commutative property of multiplication to multiply any 2 arbitrary bytes
			together.

	\subsection{Algorithm Parameters}\label{sec:aes-param}
		The FIPS 197 publication specifies three versions of AES: AES-128, AES-192, and AES-256. These versions only differ in a few parameters while the algorithm itself stays the same
		between them. In our project, we chose to implement AES-256. As such, our implementation uses keys that are 8 words long, blocks that are 4 words long, and has 14 rounds of encryption.

	\subsection{The Algorithm}\label{sec:aes-alg}
		The AES algorithm first splits the 16 byte (4 word) block into a 4x4 matrix that is referred to as the state array. The encryption is done by performing a variety of transformations on
		this matrix, which are defined below.

		\subsubsection{SubBytes Transformation}\label{sec:aes-subbytes}
			The SubBytes transformation uses a substitution box to change every byte in the state array. The substitution box is a pre-defined 16x16 matrix of bytes. Each byte in the state array
			is split into its upper 4 bits and its lower 4 bits, and is replaced by the byte in the row defined by the upper 4 bits and the column defined by the lower 4 bits. Unlike in DES,
			only one S-box is used and each byte is used by it individually.

		\subsubsection{ShiftRows Transformation}\label{sec:aes-shiftrows}
			The ShiftRows transformation performs a circular left shift on each of the rows, which each row being shifted a variable amount. The rows are indexed from $r_0$ to $r_3$ with $r_0$
			being the topmost row and $r_3$ being the bottommost row. Each row is shifted by it's index, so $r_0$ remains stationary while $r_2$ is shifted by 2 places.

		\subsubsection{MixColumns Transformation}\label{sec:aes-mixcols}
			The MixColumns transformation works on individual columns at a time, treating them as a four-term polynomial and multiplying them by a fixed polynomial 
			$a(x)=\text{03}x^3+\text{01}x^2+\text{01}x+\text{02}$, with the multiplication happening mod $x^4+1$. If we label the bytes in the column as $b_0,b_1,b_2,b_3$ where $b_0$ is in
			the topmost row and $b_3$ is in the bottommost row, we can represent the multiplication by the following equations, using the mathematical definitions from section \ref{sec:aes-math}:
			\begin{align*}
				b_0 &= (\text{0x02}\bullet b_0) \oplus (\text{0x03}\bullet b_1) \oplus b_2 \oplus b_3\\
				b_1 &= b_0 \oplus (\text{0x02}\bullet b_1) \oplus (\text{0x03}\bullet b_2) \oplus b_3\\
				b_2 &= b_0 \oplus b_1 \oplus (\text{0x02}\bullet b_2) \oplus (\text{0x03}\bullet b_3)\\
				b_3 &= (\text{0x03}\bullet b_0) \oplus b_1 \oplus b_2 \oplus (\text{0x02}\bullet b_3)
			\end{align*}   

		\subsubsection{AddRoundKey Transformation}\label{sec:aes-addround}
			The AES key is expanded into various round keys, as is defined in section \ref{sec:aes-keys}. Each of these round keys is 16 bytes, which is made of 4 different 4 4-byte chunks.
			Each of these chunks is labelled $w_0$ to $w_3$, with $w_0$ being the leftmost and $w_3$ being the rightmost. In this transformation, each of these chunks is XORed with its corresponding
			column in the state array to create the transformed state.
			
		\subsubsection{Key Expansion}\label{sec:aes-keys}
			AES-256 takes as input a 256 bit key. This key is then expanded into a number of words equal to $4(N_r+1)$ where $N_r$ is the number of rounds. For AES-256, $N_r=14$, so the key is
			expanded from 256 bits to 60 words, or 1920 bits. This is done through a combination of utilizing the S-box defined in section \ref{sec:aes-subbytes}, circularly left-shifting bytes
			similarly to section \ref{sec:aes-shiftrows}, and utilizing a round constant $rcon$ defined by $rcon[i]=\{x^{i-1}, \text{0x00}, \text{0x00}, \text{0x00}\}$. The full algorithm is
			shown in \href{https://nvlpubs.nist.gov/nistpubs/FIPS/NIST.FIPS.197.pdf}{FIPS 197}.

	\subsection{Encryption}\label{sec:aes-encryption}
		To begin the encryption process, the message is split into 16-byte blocks to be encrypted seperately. We chose to implement a CBC version of AES, so the plaintext of each block is XORed with
		the ciphertext of the previous block before encrypting.\\

		When encrypting a block, it is first transformed into its corresponding state array and an initial AddRoundKey is performed. From there, 14 rounds take place, where each round consists of
		a SubBytes, a ShiftRows, a MixColumns, and an AddRounKey in that order. The only exception is the final round, where the MixColumns transorm is skipped. At this point, the resultant state
		array can be unpacked and returned as the ciphertext.

	\subsection{Decryption}\label{sec:aes-decryption}
		Each of the transformations used in encryption can be easily inverted. For SubBytes, an inverse S-box exists that, when used the same way, results in the exact inverse of the original. For
		ShiftRows, each row is circularly shifted to the right by the same amount it was during encryption. AddRoundKey consists of only XOR operations, and as such is its own inverse. The
		MixColumns transformation can be inverted in the same method that it was performed, but using a different set of computations. The round keys can be generated in the same way as they are
		for encryption and then simply used in reverse order.\\

		When decrypting a block, it is first transformed into its corresponding state array and an initial AddRoundKey is performed. From there, each of the 14 rounds consist of inverting ShiftRows,
		inverting SubBytes, an AddRoundKey, and inverting MixColumns. The only exception once again is the last round, where inverting MixColumns is skipped.

	\subsection{Implementation}\label{sec:aes-implementation}
		Unlike the public key systems previously mentioned, AES does not really have pitfalls to avoid during implementation because all variables are pre-defined. Additionally, 
		\href{https://nvlpubs.nist.gov/nistpubs/FIPS/NIST.FIPS.197.pdf}{FIPS 197} provides a step by step example of key scheduling in Appendix A and a full encryption example in Appendix C. Both
		of these were extremely helpful for ensuring that my implementation worked properly.
